\documentclass{article}
\usepackage{fullpage}
%\usepackage{times}
\usepackage{url}
\usepackage{amsmath,amsthm,amssymb,color}
%\usepackage{multicol}
\usepackage{float}
\usepackage{graphicx}
\usepackage{hyperref}
\usepackage{listings}
%\usepackage[all]{hypcap}

\begin{document}

\title{\bf Range Analysis for the IonMonkey JavaScript engine}

\author{
Ryan Pearl\\
\texttt{$<$rpearl@andrew.cmu.edu$>$}\\
\and
Michael Sullivan\\
\texttt{$<$mjsulliv@cs.cmu.edu$>$}\\
}

\maketitle

\section{Introduction}
%% XXX: FIXME, TODO
We propose implementing and evaluating range analysis for IonMonkey,
an optimizing JIT compiler for Javascript. Range analysis is
particularly useful in this context: in addition to taking advantage
of the analysis results to eliminate redundant comparisons and
bounds-checks, Javascript's dynamic semantics requires that all
numeric computations are floating point arithmetic.  This is slow, so
numbers that can be stored as integers are, and arithmetic operations
are guarded against overflow into their floating point
representation. With range analysis, we can prove these guards
unnecessary and emit higher-quality code.


\section{Related Work}
%% XXX: FIXME, TODO
We have looked at a few papers on range analysis, such as ``The Design
and Implementation of a Non-Iterative Range Analysis Algorithm on a
Production Compiler'', by Teixeira and Pereira, which is a practical
implementation of an algorithm designed by by Su and Wagner
\cite{Su04aclass}. The algorithm due to Su and Wagner is
non-iterative, and works by solving a constraint graph. Other range
analyses, generally formulated as a more traditional iterative
dataflow problem, are also available.

We reviewed the literature on different types of range analysis and
decided to implement an analysis based on Gough and Klaeren's SSA
based range analysis algorithm \cite{Gough94eliminatingrange}.

\section{Implementation}
%% XXX: TODO

\section{Evaluation}
%% XXX: TODO

\section{Surprises and Lessons Learned}
%% XXX: TODO

\section{Conclusions}
%% XXX: TODO
Boy, JS is a pain in the ass.

\bibliography{citations}{}
\bibliographystyle{abbrv}


\end{document}
