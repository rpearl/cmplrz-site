\pagenumbering{gobble}
\documentclass{slides}
\usepackage{times}



\usepackage{amsmath,amsthm,amssymb,color}
%\usepackage{multicol}
\usepackage{graphicx}
\usepackage{hyperref}
%\usepackage{courier}
\usepackage{listings}
\usepackage{color}
%\usepackage[all]{hypcap}

\newcommand{\Squash}{\mathsf{Squash}}
\newcommand{\Clamp}{\mathsf{Clamp}}
\newcommand{\Range}{\mathsf{Range}}

\let\x\cdot


\begin{document}

\title{\bf Range Analysis for the IonMonkey JavaScript Compiler}
\author{Ryan Pearl and Michael Sullivan}

\maketitle

\begin{slide}
\begin{center}
{\Huge \bf{IonMonkey}}
\end{center}

\begin{itemize}
\item IonMonkey is Mozilla's next generation Javascript engine
\item Optimizing JIT compiler
\item Modern SSA-based compiler with internals inspired by LLVM
\item Performs optimistic type specialization and bails out when type
  assumptions fail
\item Designed to be easy to extend
\end{itemize}
\end{slide}

\begin{slide}
\begin{center}
{\Huge \bf{JavaScript arithmetic}}
\end{center}
\begin{itemize}
\item JavaScript numbers defined to be double precision floating point
\item For efficiency, implementations store and operate on them as
  integers whenever possible
\item IonMonkey will generates type specialized code for integer values
\item If arithmetic overflows, need to bail out of integer code and
  promote to doubles
\item Range analysis could eliminate these overflow checks
\end{itemize}
\end{slide}


\end{document}
