\pagenumbering{gobble}
\documentclass{slides}
\usepackage{times}



\usepackage{amsmath,amsthm,amssymb,color}
%\usepackage{multicol}
\usepackage{graphicx}
\usepackage{fullpage}
\usepackage{hyperref}
%\usepackage{courier}
\usepackage{listings}
\usepackage{color}
%\usepackage[all]{hypcap}

\newcommand{\Squash}{\mathsf{Squash}}
\newcommand{\Clamp}{\mathsf{Clamp}}
\newcommand{\Range}{\mathsf{Range}}

\let\x\cdot
\lstset{
  language=Python,
  showstringspaces=false,
  tabsize=4,
  basicstyle=\ttfamily,
  keywordstyle=\bfseries\ttfamily,
  captionpos=b
}
\lstdefinelanguage{JavaScript}{
  keywords={if,else,var},
  keywordstyle=\bfseries\ttfamily,
  sensitive=false,
  comment=[l]{//},
  morecomment=[s]{/*}{*/}
}


\begin{document}

\title{\bf Range Analysis for the IonMonkey JavaScript Compiler}
\author{Ryan Pearl and Michael Sullivan}

\maketitle

\begin{slide}
\begin{center}
{\Huge \bf{IonMonkey}}
\end{center}

\begin{itemize}
\item IonMonkey is Mozilla's next generation Javascript engine
\item Optimizing JIT compiler
\item Modern SSA-based compiler with internals inspired by LLVM
\item Performs optimistic type specialization and bails out when type
  assumptions fail
\item Designed to be easy to extend
\end{itemize}
\end{slide}

\begin{slide}
\begin{center}
{\Huge \bf{JavaScript Arithmetic}}
\end{center}
\begin{itemize}
\item JavaScript numbers defined to be double precision floating point
\item For efficiency, implementations store and operate on them as
  integers whenever possible
\item IonMonkey will generates type specialized code for integer values
\item If arithmetic overflows, need to bail out of integer code and
  promote to doubles
\item Range analysis could eliminate these overflow checks
\end{itemize}
\end{slide}

\begin{slide}
\begin{center}
{\Huge \bf{Range Analysis}}
\end{center}
\begin{itemize}
\item Associate each integer temporary with a range: a contiguous subset of $\{-\infty\} \cup [-2^{31}, 2^{31}-1] \cup \{\infty\}$
\item Can do arithmetic on ranges:
\end{itemize}
\begin{eqnarray*}
\Clamp(n) &=& \begin{cases}
-\infty &\text{if } n < -2^{31} \\
\infty &\text{if } n > -2^{31}-1 \\
n&\text{otherwise } \\
\end{cases}\\
\Clamp([l, h]) &=& [\Clamp(l), \Clamp(h)] \\
{[x_l, x_h] \cup [y_l, y_h]} &=& [\min(x_l, y_l), \max(x_h, y_h)] \\
{[x_l, x_h] \cap [y_l, y_h]} &=& [\max(x_l, y_l), \min(x_h, y_h)] \\
{[x_l, x_h] + [y_l, y_h]} &=& \Clamp ([x_l + y_l, x_h + y_h]) \\
{[x_l, x_h] - [y_l, y_h]} &=& \Clamp ([x_l - y_h, x_h - y_l])
\end{eqnarray*}
\end{slide}


\begin{slide}
\begin{center}
{\Huge \bf{XSA Form}}
\end{center}
\begin{itemize}
\item Want a way to track information learned from conditionals without needing to consider more than one range per temp
\item Introduce $\beta$ nodes to associate range information from conditionals with temporaries
\item Some example code with $\beta$ nodes; all of the ranges checks can be eliminated
\end{itemize}
{\tiny
\begin{verbatim}
if (x < 0) {
    x1 = Beta(x, [INT_MIN, -1]);
    y1 = x1 + 2000000000;
} else {
    x2 = Beta(x, [0, INT_MAX]);
    if (x2 < 1000000000) {
        x3 = Beta(x2, [INT_MIN, 999999999]);
        y2 = x3*2;
    } else {
        x4 = Beta(x2, [1000000000, INT_MAX]);
        y3 = x4 - 3000000000;
    }
    y4 = Phi(y2, y3);
}
y = Phi(y1, y4);
\end{verbatim}
}
\end{slide}



\end{document}
